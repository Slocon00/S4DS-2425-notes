\section{Power laws and Zipf's law}

Power laws are a family of ``scale free'' distributions. Most distributions have a typical size or scale, so they have some value around which measurements are centered. In contrast, power laws vary over a very large range where it's not possible to identify a typical value around which the distribution peaks.

\distribution{Power law distribution}{$X \sim Pow(x_{min}, \alpha)$}{
    A random variable has the power law distribution if for some $\alpha > 1$ its PDF is given by
    \begin{align*}
        &f(x) = C \cdot x^{-\alpha} &x \geq x_{min}
    \end{align*}
}
$C$ is called \textbf{intercept}, while $\alpha$ is called \textbf{exponent}. If the function is expressed in logarithmic scale, we have
\[
    \log(f(x)) = -\alpha \cdot \log(x) + \log(C)
\]
i.e., there is a linear relationship between $\log(f(x))$ and $\log(x)$. Graphically, this means that the distribution is a straight line in a log-log plot. The reason parameter $x_{min}$ is included is to specify what is the exact lower bound after which a distribution shows a power law behaviour.

Being scale-free, we can identify some constant $b$ such that $p(bx) = g(b)p(x)$, meaning that even if we multiply the variable by this scaling factor, the form of the distribution remains the same. In this case, we write
\[
    p(bx) = b^{-\alpha} C \cdot x^{-\alpha}.
\]
Notice how the value of the intercept is not specified in the definition above. This is because after fixing $x_{min}$ and $\alpha$, $C$ is uniquely determined by the condition that the integral of the PDF over the entire range must be 1:
\begin{align*}
    1 = \int_{x_{min}}^{\infty} C \cdot x^{-\alpha} \ dx = \frac{C}{-\alpha + 1}[x^{-\alpha + 1}]_{x_{min}}^{\infty} = \frac{C}{\alpha - 1} x_{min}^{-\alpha + 1} \iff \boxed{C = \frac{(\alpha - 1)}{x_{min}^{-\alpha + 1}}}
\end{align*}
This integral is finite only if $\alpha > 1$. If $\alpha < 1$, then it simply diverges. If $\alpha = 1$, the denominator becomes 0, and the integral is not defined. By substituting this value in the formula of the PDF, we get
\begin{equation*}
    f(x) = \frac{\alpha - 1}{x_{min}} \left ( \frac{x}{x_{min}} \right )^{-\alpha}.
\end{equation*}
Using the same calculations we can find a closed formula for the CCDF:
\begin{equation*}
    P(X > x) = \int_{x}^{\infty} C \cdot x^{-\alpha} \ dx = \frac{C}{-\alpha + 1}[x^{-\alpha + 1}]_{x_{min}}^{\infty} = \frac{C}{\alpha - 1} x^{-\alpha + 1}.
\end{equation*}
Since we calculated $C$ we can substitute it back in the formula to get
\begin{equation*}
    P(X > x) = \left ( \frac{x}{x_{min}} \right )^{-\alpha + 1}
\end{equation*}
Both the PDF and the CCDF have the same form, but with a different exponent. The CCDF also looks linear when plotted in a log-log scale.
As for the expectation, we have
\begin{equation*}
    \E[X] = \int_{x_{min}}^{\infty} x \cdot C \cdot x^{-\alpha} \ dx = C \int_{x_{min}}^{\infty} x^{\-alpha + 1} \ dx = \frac{C}{-\alpha + 2} \left [ x^{-\alpha + 2} \right ]_{x_{min}}^{\infty} = \frac{C}{\alpha - 2} x_{min}^{-\alpha + 2}.
\end{equation*}
Similarly to the calculations done to find $C$, we can observe how this integral is finite only for $\alpha > 2$: if $\alpha < 2$, the integral diverges, while if $\alpha = 2$, the denominator becomes 0 and the integral is not defined. Substituting the value of $C$ back in the formula, we get
\begin{equation*}
    \E[X] = \frac{\alpha - 1}{\alpha - 2} x_{min}
\end{equation*}
Also for the variance, it is finite only for $\alpha > 3$.

\distribution{Pareto distribution}{$X \sim Par(x_{min}, \beta)$}{
    A random variable has the Pareto distribution if for some $\beta > 0$ its density function is given by
    \begin{align*}
        &f(x) = C \cdot x^{-(\beta + 1)} &x \geq x_{min}
    \end{align*}
}
A Pareto distribution is actually just a power law, but expressed differently: \\
$Par(x_{min}, \beta) = Pow(x_{min}, \beta + 1)$.
\distribution{Discrete power law distribution}{$X \sim Pow(\alpha, k_{min})$}{
    A random variable has the discrete power law distribution if for some $\alpha > 1$ its PMF is given by
    \begin{align*}
        &p(k) = C \cdot k^{-\alpha} &k = k_{min}, k_{min}+1, \dots
    \end{align*}
}
Since the sum of probabilities must be 1, C is determined as
\begin{equation*}
    C = \frac{1}{sum_{k=k_{min}}^{\infty} k^{-\alpha}} = \frac{1}{\zeta(\alpha, k_{min})}
\end{equation*}
$\zeta(\alpha, k_{min})$ is the \textbf{Hurwitz zeta function}. A special case of it is the \textbf{Riemann zeta function}, which is $\zeta(\alpha) = \zeta(\alpha, 1)$

When we are studying a data sample and want to check if it follows a power law, we can plot the frequency of its values in log-log scale and verify if the points are aligned in a straight line. However, since the values in the tail are rarer, the data sample will have few of them. The resulting plot will likely present a lot of noise around the tail of the distribution, and it may not be obvious whether it is a power law or some similar distribution (such as exponential or log-normal). To fix this issue, we can follow two approaches:
\begin{itemize}
    \item We estimate and plot the CCDF in log-log scale. As mentioned above, the CCDF of a power law also appears linear when plotted this way, with the advantage of being much more stable in the tail.
    \item We construct an histogram using logarithmic binning. This means that the bins are not equally spaced, but they grow exponentially. For example, the first bin goes from 1 to 10, the second from 10 to 100, the third from 100 to 1000, and so on. Since bins aggregate values, the effect of noise is reduced.
\end{itemize}

\distribution{Zipf's law distribution}{$X \sim Zipf(\alpha)$}{
    A random variable has the Zipf's law distribution if for some $\alpha > 1$ its PMF is given by
    \begin{align*}
        &p(r) = C \cdot r^{-\alpha} &r = 1, 2, \dots, N
    \end{align*}
}
In a Zipf's law distribution, probabilities are assigned to the \textbf{ranks} of events; the higher the rank (i.e. closer to 1), the higher the probability. This is different than a power law distribution, where probabilities directly depend on the frequencies. For example: a discrete power law may model the probability of a word with a certain number of accurrences in a text, while Zipf's law may model the probability of a word with a certain rank in a list of words sorted by frequency.

We can try to convert a power-law into a Zipf's law and vice-versa. By comparing the PMF of a Zipf's law and the CCDF of a power law, they have the same form, and are actually representing the same information but with the axes inverted:

\begin{figure*}[ht]
    \centering
    \includegraphics*[width=0.8\textwidth]{img/zipf_vs_ccdf.png}
\end{figure*}
The rank $r$ of a word with frequency $k$ is equal to the number of words with frequency larger than $k$ plus 1. In other words, $r = 1 + N \cdot P(X > k)$. If $X \sim Pow(1, \alpha)$, then $r = 1 + N \cdot P(X > k) \propto k^{-{(\alpha - 1)}}$. By inverting, we get that $k \propto r^{-\frac{1}{\alpha - 1}}$, i.e., the frequencies are Zipf's law distributed with parameter $\frac{1}{\alpha - 1}$.
\[
    X \sim Pow(1, \alpha) \iff R \sim Zipf \left (\frac{1}{\alpha - 1} \right )
\]
($R$ is a r.v. that models the ranks).\\
For this distribution, $C$ is calculated as
\begin{equation*}
    C = \frac{1}{\sum_{r=1}^{N} r^{-\alpha}} = \frac{1}{\zeta(\alpha) - \zeta(\alpha, N+1)}
\end{equation*}