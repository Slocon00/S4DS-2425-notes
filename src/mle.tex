\section{Maximum Likelihood Estimation}

The previous sections showed different ways to derive parameter estimators using the ``plug-in method'': knowing a sample, we use a formula of random variables, substitute the empirical data, and obtain an estimate. This section will introduce a more general parametric principle to derive estimators, called \textbf{Maximum Likelihood Estimation} (\textbf{MLE}). The maximum likelihood principle states: ``Given a dataset, choose the parameter(s) of interest that maximize the likelihood of observing that dataset''. 

\boxdefinition{Likelihood, Log-likelihood}{
    Let $x_1, \ldots, x_n$ be a dataset, realization of a random sample $X_1, \ldots, X_n$, where the PMF/PDF of $X_i$ is $f_{\theta}()$, parametric on $\theta$. The likelihood function is defined as
    \[
        L(\theta) = \prod_{i=1}^n f_{\theta}(x_i)
    \]
    and the log likelihood function is defined as
    \[
        l(\theta) = \log L(\theta) = \sum_{i=1}^n \log f_{\theta}(x_i)
    \]
}
The goal of MLE is to find the parameter value(s) that maximize the likelihood of the data. Since finding the maximum (in practice, derivating) a product is difficult, the logarithm of it is used instead, since it is a monotonic function (so the maximum remains the same) and the original likelihood can be converted to a sum of terms. Another issue to consider when using products is that when it is computed, the result can become very small, leading to numerical instability.
\boxdefinition{Maximum Likelihood estimates}{
    The maximum likelihood estimates of $\theta$ is the value $t = \arg\max_{\theta} L(\theta) = \arg\max_{\theta} l(\theta)$. The statistics over the random sample
    \[
        \hat{\theta}_{ML} = \arg\max_{\theta} L(\theta) = \arg\max_{\theta} l(\theta)
    \]
    is called maximum likelihood estimator for $\theta$.
}
Often, loss functions are used to evaluate the quality of the estimator. Some examples are:
\begin{itemize}
    \item \textbf{Negative log-likelihood}: $nLL(\theta) = -l(\theta)$
    \item \textbf{Akaike Information Criterion} (\textbf{AIC}): $AIC = 2 |\theta| - 2 l(\theta)$. This function balances model fit (in the second term) with model simplicity (in the first term).
    \item \textbf{Bayesian Information Criterion} (\textbf{BIC}): $BIC = |\theta| \log(\theta) - 2 l(\theta)$. It has a stronger penalty for model complexity compared to AIC.
\end{itemize}
The last two can be used to compare models with different number of parameters since they consider complexity as well as fit.

MLE estimators can be biased, but under mild assumptions, they are always asymtotically unbiased, and (also asymptotically) they always have the smallest variance among unbiased estimators. Additionally, if $\hat{\theta}$ is the MLE estimator of $\theta$ and $g()$ is an invertible function, $g(\hat{\theta})$ is the MLE estimator of $g(\theta)$.

\paragraph{Cross entropy and negative log likelihood} There is a connection between the cross entropy and the negative log likelihood functions. Assume we have the random variables $X,Y$ with PMF $p_x$ and $p_y$. The cross entropy of $X$ with respect to $Y$ is defined as
\[
    H(X;Y) = - \sum_i p_x(a_i) \log p_y(a_i).
\]
The negative log-likelihood is
\[
    nLL(\theta) = - \sum_i \underset{\textit{This is } \log p_y(a_i)}{\underbrace{\log(f_{\theta}(x_i))}} = H(X;Y)
\]
where $X \sim F_n$ and $Y \sim F_{\theta}$. Minimizing the cross entropy (or the KL-divergence) between empirical and theoretical distributions is equivalent to maximizing the likelihood of the data.

\paragraph{Score function and Fisher Information}
The entropy of a random variable is the mean information carried by it. Then, the partial derivative (w.r.t. $\theta$) of the logarithm of $f_{\theta}$ represents the change in information as $\theta$ varies. It turns out that
\[
    \E\left[\frac{\partial}{\partial \theta} \log f_{\theta}(X)\right] = 0
\]
for any $X$. Instead of studying the expectation, we consider the variance. Formally, we define the score function and the Fisher information as follows:
\boxdefinition{Score function}{
    The score function is the random variable
    \[
        S(\theta) = \frac{\partial}{\partial \theta} l(\theta) = \sum_{i=1}^n \frac{\partial}{\partial \theta} \log f_{\theta}(X_i)
    \]
}
\boxdefinition{Fisher information}{
    The Fisher information is the variance of th score function:
    \[
        I(\theta) = Var(S(\theta)) = \E[S(\theta^2)]
    \]
}
The Fisher information measures the sensitivity of $X$ with respect to $\theta$. If small changes in $\theta$ cause large changes in the density function, the Fisher information is high: this means that the data provides information on the correct $\theta$.

For unbiased estimators, the \textbf{Cramér-Rao bound} holds, stating that for any unbiased estimator $T$:
\[
    Var(T) \geq \frac{1}{I(\theta)}
\]
An unbiased estimator for which the equality holds is a Minimum Variance Unbiased Estimator. The \textbf{absolute efficiency} of an estimator can be then calculated as
\[
    e(T) = \frac{1}{I(\theta) \cdot Var(T)} \in [0,1]
\]
Recall that a MLE estimator is always asymptotically unbiased and has asymptotic minimum variance. Also by Cramér-Rao, asymptotically we have that
\[
    se(\hat{\theta}_{ML}) = \sqrt{Var(\hat{\theta}_{ML})} = \frac{1}{\sqrt{n \cdot I(\theta)}}
\]
where $se$ is the standard error of the estimator.